%!TEX root = ../vrrelint.tex
% Data Collection
% 500 Words

To attempt to capture the lived experience of the participants, and following the principles of phenomenology, I interviewed the participants one-to-one and attempted to lead the conversation towards how they experienced intimate relationships in \vrc.

The interview followed a semi-structured interview, with the intent of guiding the participant to describe and talk about instances of intimacy in \gls{vr}, and how they experienced them, as well as their views on the generic topic of intimacy in the game. The questions and prompts I followed to direct the interview were as follows:
\begin{enumerate}
	\item\label{interview::covert::motivation} What got you interested in playing \vrc? Has that motivation changed?
	\item\label{interview::covert::interactions} Do you enjoy interacting with people in \vrc? Could you talk about memorable interactions? 
	\item\label{interview::covert::friendships} Did you make friends or establish any kind of relationship?
	\item\label{interview::overt::descibemoment} Describe a moment in which you experienced intimacy in \vrc.
	\item\label{interview::overt::vrvsonline} Did the \gls{vr} medium change the experience compared other online mediums? 
	\item\label{interview::overt::feelaboutintimacy} How do you feel about intimacy and intimate relationships in \gls{vr}?
\end{enumerate}

The objective of these lines of questioning was focusing the interview of the topic of relationships and intimacy and reduce the number of irrelevant outcomes.

Priming or influencing the participants was a concern, therefore I decided to split the interview in two parts.
In the first part, consisting of $3$ indirect prompts or lines of questioning, to mitigate the risk of influencing or priming the subject, I avoided asking the research question directly, instead asking about motivations (Question~\ref{interview::covert::motivation}), interactions (Question~\ref{interview::covert::interactions}), and developed relationships (Question~\ref{interview::covert::friendships}).
The rationale being that this line of conversation might lead the participants to talk about their relationships, how much these motivated them to spend time in \gls{vr} and how they lead to intimacy and their experience.

After this covert part of the interview, I asked some more overt questions, to focus even further, and learn how they experienced intimacy in \gls{vr} and their opinion on the topic.
In particular, following the typical phenomenological approach seen in the literature~\cite{register92}, the second part starts with an open-ended (Question~\ref{interview::overt::descibemoment}) request to describe a moment of intimacy in \vrc. 
While this is more direct, and reveals the objectives of the interview, it also leads directly to the description of relevant experiences.
Even though this second part is subject to the researcher's influence, this can be taken into account during the analysis, while still providing useful information on the participants lived experiences.

Video and audio data was collected from 5 interviews, totaling \todo{sum runtime} minutes of video. 
Participants were selected by approaching players in \vrc and asking them if they were open to being interviewed and consented to recording.