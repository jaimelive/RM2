%!TEX root = ../vrrelint.tex
% Introduction
% 500 Words

Increasing advancements in \gls{vr} technology and its lowering price point, created a new major medium for social interactions, and a new experience of relationships and intimacy.
In context of the COVID-19 pandemic of the early 2020's, and the resulting quarantine and decreased in-person socialization opportunities, online social experiences / games saw an increased influx of new users, attempting to stave off loneliness.
Mediums used include, among others, text chat, \gls{voip} calls, video calls and \gls{vr}. 

Online games and experiences, in particular in \gls{vr}, became the de facto socialization medium for an unprecedented number of people, in particular younger demographics.
Compared to other online mediums, \Gls{vr} increases immersion as a result of features such as:
\begin{enumerate}
	\item Stereoscopic image;
	\item Full coverage of the users field of view;
	\item High quality positional audio;
	\item Haptic gear, from controllers to full body actuators to simulate touch;
	\item Differing degrees of body tracking (starting at 3 point tracking for head and hands, extending to 7 point full body tracking).
\end{enumerate} 
However, it still misses the possibility of physicality of real world interactions, which might lead to an incomplete or possibly frustrating experience.
Furthermore, the effects of experiencing intimacy in this context are not well known, and might have a negative effect, possibly of greater impact if they replace socialization at early developmental stages.

The literature recognizes these changes~\cite{power2020new} and there is some research towards determining if these are positive or negative as well as how it will affect human interaction~\cite{essig2018technology,Kanwal2019}, going as far as to consider the possibility of a dystopian future where \gls{vr} and \gls{ai} might come to replace intimacy for most people.

A growing number of users, increasingly spending more time socializing in \gls{vr}, as well as the distinct nature of these experiences (\eg embodied, immersive) motivates this work, and future study in how people experience relationships and intimacy in this new context.

The starting point is to attempt to extract a common meaning or essence by collecting data on the lived intimacy experiences of a sample of players. 
Consequently, I focused on a \gls{vr} game, \vrc, and conducted a phenomenological study on the nature of intimacy in \gls{vr}.

The rest of the paper is organized as follows:
Section~\ref{sec::soa} gives a quick overview of work already done on the topic.
Section~\ref{sec::methodology} presents the methods employed for sampling, research and data collection.
Section~\ref{sec::analysis} describes the analysis procedure and presents its results.
Section~\ref{sec::discussion} discusses the results and extracted shared topics and meaning.
Section~\ref{sec::interpretationlimits} describes the limitations of the method and results.
Finally, Section~\ref{sec::conclusions} concludes the work and lays out future work.