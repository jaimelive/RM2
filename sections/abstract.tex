%!TEX root = ../vrrelint.tex

% First stage - Theme proposals - to refine - by end of April
% Answer the following questions as if you were writing an abstract (300 words):

% - Paradigm and Methodological Approach?
% - Data Sources?
% - Analysis procedure?
% - Expected Results?

% Relationships and Intimacy in Virtual Reality Social Settings

% - Main Question?
As a result of the reduced entry-cost barrier, and increased hardware availability, \Gls{vr} social experiences are becoming more popular and gathering growing numbers of users.
This trend was further exacerbated by the social deprivation caused by quarantine and reduced social opportunities during the last few years. 
From observation, and the existing literature, there are new social experiences and phenomenons that are unique to this new medium, that are currently not well understood, namely how \gls{vr} experiences affect the way users establish and experience relationships and intimacy. 
% - Paradigm and Methodological Approach?
Following a phenomenological approach, I study and describe how \gls{vr} users experience relationships and intimacy in this new context, and how this is affected by it's unique aspects, such as anonymity, global reach and different degrees of embodiment.
% - Data Sources?
Data was collected by recording both casual conversation and structured interviews, conducted in the social \gls{vr} game \emph{VRChat}.
Recording video, preserves the body language of the participants, making it possible to take embodied experiences and their effects into consideration. 
% - Analysis procedure?
The transcribed data, annotated with body language, was analyzed to extract recurring, significant statements and experiences.
From there, more structured descriptions of relationships and intimacy experiences were created, paying attention to how the addition of embodiment affected them. It was of particular interest, how different levels of embodiment affected the experience (\eg head and hand tracking or full body tracking).
% - Expected Results?
Finally, I present the descriptions of the participants' common experiences with relationships and intimacy in the context of \gls{vr} and how this new unique medium affected them as well as how the author perceived them.
These results can be used to advance the state of the art in \gls{vr} social games, generating more positive outcomes for both the users and developers.




% Generate themes from the analysis of significant statements. Phenomenological data analysis steps are generally similar for all psychological phenomenologists who discuss the methods (Moustakas, 1994; Polkinghorne, 1989). Building on the data from the first and second research questions, data analysts go through the data (e.g., interview transcriptions) and highlight “significant statements,” sentences, or quotes that provide an understanding of how the participants experienced the phenomenon. Moustakas (1994) calls this step horizonalization. Next, the researcher develops clusters of meaning from these significant statements into themes.
% Develop textural and structural descriptions. The significant statements and themes are then used to write a description of what the participants experienced (textural description). They are also used to write a description of the context or setting that influenced how the participants experienced the phenomenon, called structural description or imaginative variation. Moustakas (1994) adds a further step: Researchers also write about their own experiences and the context and situations that have influenced their experiences. We like to shorten Moustakas’s procedures and reflect these personal statements at the beginning of the phenomenology or include them in a methods discussion of the role of the researcher (Marshall & Rossman, 2015).