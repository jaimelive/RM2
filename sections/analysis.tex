%!TEX root = ../vrrelint.tex
% Analysis
% 500 Words

To begin, in the spirit of the phenomenological approach, the researcher must, to the best of his ability, set aside personal beliefs and preconceptions (\ie bracketing).
Afterwards, the data was analyzed using the \gls{sck} method of analysis with \citeauthor{moustakas1994phenomenological}~\cite{moustakas1994phenomenological}'s modifications as simplified by \citeauthor{creswell2016qualitative}~\cite[Chapter~8]{creswell2016qualitative}

In a simplified manner, the method involves the following steps:
\begin{enumerate}
	\item List statements that are significant to the studied experience as well as non-overlapping nor repetitive, and treat them as having equal worth;
	\item Group significant statements into themes (also called meaning units);
	\item Describe, with inclusion of verbatim example, ``what'' the participants experienced with the phenomenon (textural description).
	\item Describe ``how'' the experience happened, exposing the context in which the phenomenon took place (structural description);
	\item Composite a description of the phenomenon that relates the ``essence'' of the experience, with focus on ``what'' and ``how'' the participants experienced it. 
\end{enumerate}

\subsection{Results}
After the first stage, without trying to reduce too much to void losing detail, I could identify $90$ significant statements.
Preliminary grouping yielded $13$ pseudo-themes that could be further abstracted in $5$ themes or meaning units.
Table~\ref{tab:themes} summarizes themes.

\begin{table}[t]
  \caption{Themes and summarized description}
  \label{tab:themes}
  \begin{tabularx}{\textwidth}{lX}
    \toprule
    Theme&Description\\
    \midrule
     Anxiety and Appearance Dysmorphia & Players feel anxious, both in game and in real life. Some use it to practice social interaction. VR seems to relate to appearance dysmorphia.\\
     Presence, Embodiment and Doing Things & There is a feeling of being in the game, as well as feeling the presence of other people, participating in embodied activities together.\\
     Socialization and Friends & Socialization, friendships and mutual support develop, and are a motivator for playing the game.\\
     Romance & It's common to develop romantic relationships. They feel more realistic than the normal long distance relationship, because of Presence and Embodiment.\\
     Touch and Intimacy & People pet and hug each other and participate in physical intimacy and sex simulation. Some describe the notion of \emph{Phantom Touch}, where they can seem to feel simulated touch.\\
  \bottomrule
\end{tabularx}
\end{table}

% Describe the essence of the theme.
% Textural decsription: How did it happen 
% Structural decsription: How did it happen (context)

\subsubsection{Theme 1: Anxiety and Appearance Dysmorphia}\label{theme:anxiety_dysmorphia}
It's common to hear descriptions of social anxiety and appearance dysmorphia.
Being able to choose or even create their own avatar, seems to be experienced positively.
\say{... probably everyone I've ever met, including myself. They don't really like they way they are in real life, so they come to here in VR to become the person they wanted to be ... either one gender or the other, one body type or the other, its just how they wanna be}.


Another participant said it helped them adjust their appearance to their self-image,
\say{You know, it's me, my usual self, but different body, different everything. And it made me feel great}.
For participants with low confidence, it seemed to create a way to establish relationships without having to expose their looks, someone said,
\say{people might be scared of showing their real self, because like... you don't wanna mess this up [possible relationship] showing yourself. Like, me myself, I have some fat... but I don't care that much... but like, some people will judge that and choose not to be your friend because of how you look}.
All participants experienced in a positive way, going as far as to say,
\say{I just felt happier here than anywhere else I've been [pause] for once.}

There is also notion that \gls{vr} can be used therapeutically, \say{I still think if it wasn't for me hopping in that server and becoming slightly popular for a while that I wouldn't end up here because I would've probably still been in my shell of \emph{antisocialness}} and \say{... it kinda helps with me, when it comes to my anxiety}.
A young college aged male said, \say{I like meeting new people, I might be scared of meeting new people}, hinting at using it as a replacement for social interaction.
In extreme cases, some people choose to not talk at first, \say{They don't like their voice. Lack of confidence in speech or just don't want to talk.}

Adaptation leads to new behavior and normalization of things initially perceived as uncommon.
\say{It's weird, you jump into this game, and all these weird things become normal, you see them every day and you do them as well}.

A participant who called themselves \emph{mirror dweller}, said, \say{When I got the VR Headset, the game became a completely new experience, I started petting people and sitting by the mirror as well.}, \say{the first thing I did was go to a mirror cause I wanted to observe myself and see this new me}.

\subsubsection{Theme 2: Presence, Embodiment and Doing Things}\label{theme:presence_embodiment}
Immersion and a feeling of embodiment and presence was the most common theme. 
\ie
\say{So when it comes to VR you get to be there...}, 
\say{It's like going out and meeting these people in person.},
\say{ you're there, your body is moving, you see it with your own eyes, you're in that world like if you were in real life}.
Temporarily playing without \gls{vr} seems to be received with dissatisfaction, as show by: \say{... especially on PC you don't even have hands or can't really turn your head}, \say{... on PC you can't really do much, and same thing just being on a call... you can't really do anything, you're just on the other side}.

Participants describe an experience of realness and of being in a real world.
When comparing \gls{vr} to playing on a computer, someone said, 
\say{[playing on PC] but you can't really hug or give high fives or show that full expression of yourself}, 
\say{I felt like I was there [when using VR]. Everything just felt like a real world experience}.
This was noted to increase the feeling of intimacy and realness of relationships as well,
\say{I mean... VR is... VR is really a big factor when you're having a relationship in this game. I mean, sure, you can play on desktop, you can do that, but it's not the same thing. When you are in VR you get a sense of I'm so close to this person, yet so far away, but it doesn't matter because they're right there in a sense.}

Recognition of the incompleteness of the experience is borught up multiple times, 
\say{You feel like the person is there, but they are actually not there at the same time}.
The idea of \emph{they are there, but not there} is almost universal when describing the experience.

Observing themselves in mirrors, seems to increase the sense of embodiment, 
\say{... being a mirror dweller ... people like me either wanted to observe themselves}, 
resulting in a high prevalence of \emph{mirror dweller}, especially people who are not only in \gls{vr} but are using full-body tracking, meaning their movements are being captured and reproduced by their avatar, including limbs. 

\subsubsection{Theme 3: Socialization and Friends}\label{theme:socialization}
Friendship and socialization are valued by all participants.
\say{[Motivation to play] Mostly friends. And just having conversations with random people every day}
\Gls{vr} is seen as the best way to socialize, 
\say{Yea, I definitely say VR is, again my opinion, the best way to socialize. If you can't go out of your house because of COVID or you actually have a problem that keeps you home, then VR is one step to improving through it all and actually still socializing in the end}.

People have fond stories of how they met their current groups of friends as well as a sense that it might not have been possible without \gls{vr},
\say{[First VRChat friends] we were playing Murder 4, first time hopping on and the first guy I met on there, he was doing a goofy impression ... it was hilarious and I joined in, and then he introduced me to his friends who were still in the world. So it just became this big group of people.},
\say{I was the popular one for a while, which was odd, and then I just became one of them}.
One participants went as far as saying, 
\say{They are better [compared to real life friends]! They often have the same interests as me}.
This supports that the distinction between real life and virtual friends is thin, as a participant puts it, 
\say{.. it's kind of real life kinda, you can enter a place meet new different people}.

There is also a narrative of support, particularly emotional,
\say{... I greet that girl before she talked to me. She's like "My whole life is shit I want to tell someone about this now" And she started crying. The whole lobby came here saying that she's not the only one who's fucked} as well as a sense of accomplishment from helping friends,  
\say{... there was a guy who was gonna serve a prison sentence ... for 2.5 years. I just told him: I is what it is, you got caught doing stupid shit. You just gotta do your time. ... He's doing really well now. He's clean, he's got a job and stuff. I helped him as much as I could.}

\subsubsection{Theme 4: Romance}\label{theme:romance}
Related to the previous theme, most of the participants have experienced romantic feelings in \gls{vr} and have or had long-distance relationships in \vrc.
\say{Yes [got a girlfriend in VRChat], but virtual only. We don't have plans for getting it real},
\say{... this year, I got a girlfriend in here}
In general, \gls{vr} is perceived to improve the long-distance relationship, 
\say{It's a weird topic when it comes to that [long distance relationships], but it feels better in VR}.
Some people seem pretty happy about them, and describe how they will or are currently switching to a physical relationship, 
\say{I'm going to meet her soon again. She lives in Sweden, I will drive 14 hours. I will make it}.

By some accounts, there are hardships associated with this type of relationship, one participant said, 
\say{I mean, I've had a couple of intimate relationships [in VRChat], some girlfriends},
\say{At first it was kinda hard, cause of the long distance relationship. But after a while you just got used to it},
\say{... you know, time zones... we both had jobs and stuff like that, so we just kinda drifter apart}.

The presence, embodiment and simulated touch are seen as advantages,
\say{Yea, people do that a lot [going to sleep together in VR], I do that every day. It feels like you're with the person that you like. Like they're with you. You have a bed pillow, and they feel like they're actually there, but they're not there}, \say{... in VR you don't feel like its a long distance relationship because you feel like you're actually there}.
At the same time they seem to recognize that something is missing in the experience, bring back again the idea of \emph{they are there, but not there} from Theme~\ref{theme:presence_embodiment}.

Negative emotions and bad outcomes were also described, 
\say{... me and my friends were hanging out and one of their girlfriends at the time, broke up with them on the spot, like in front of us all}.

\subsubsection{Theme 5: Touch and Intimacy}\label{theme:touch_intimacy}
% Textural
\Gls{vr} seems to shine when it comes to simulated touch and feeling intimate.
Descriptions of intimate touch simulation were common, including sexual simulation, 
\say{I've seen people get cuddly and make out and stuff in VR},
\say{like sleeping together and looking to them next to you},
\say{... unless they are doing sex in VR than that's kinda like... that's.. they should be doing that somewhere else},
\say{ Oh Yea, Yea. You see that [people getting physical] everywhere you go. I mean, I have done it to people. Yes, but that's only because they like that sort of touchy feely kind of thing}.

There are accounts of a phenomenon called \emph{Phantom Sense} where the user of \gls{vr} can feel the virtual touch they see,
\say{I personally don't feel phantom sense. I know a few people who do, but every-time I ask them about it, they tell me how it's either like a light cold touch or warm even, or others will say it's like a sharp pain, especially if they got, like, shot in VR},
\say{Feels bizarre [being touched in VR], it's like ASMR but on your head}.

People spend hundreds to thousands of dollars to acquire the equipment for full-body tracking to feel even more intimate and embodied, and people who have it seem to be seen as being of higher status, 
\say{... she [girlfriend] has VR and Full Body Tracking},
\say{... actually showing real affection kind of. It's like a weird wavelength... thing... like you feel like you're there and you can touch them}.

All these effects lead to a sense of affection and intimacy, say{People usually be very emotional and honest when it's midnight so I could get info from them.},
and of actually knowing the person,
\say{[what it felt like meeting girlfriend in person] ... like I knew her already. But I found some new parts}.




