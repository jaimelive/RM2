%!TEX root = ../vrrelint.tex
% Analysis
% 500 Words

Afterwards, the data was analyzed using the \gls{sck} method of analysis with \citeauthor{moustakas1994phenomenological}~\cite{moustakas1994phenomenological}'s modifications as simplified by \citeauthor{creswell2016qualitative}~\cite{creswell2016qualitative}
In a simplified manner, in involves the following steps:
\begin{enumerate}
	\item List statements that are significant to the studied experience as well as non-overlapping nor repetitive, and treat them as having equal worth;
	\item Group significant statements into themes (also called meaning units);
	\item Describe, with inclusion of verbatim example, ``what'' the participants experienced with the phenomenon (textural description).
	\item Describe ``how'' the experience happened, exposing the context in which the phenomenon took place (structural description);
	\item Composite a description of the phenomenon that relates the ``essence'' of the experience, with focus on ``what'' and ``how'' the participants experienced it. 
\end{enumerate}
